\section{Discussion}
Mostly to be written by Brian
\subsection{Distributions}

\begin{itemize}
	\item Bounding box away from $0$ and $1$ means muscle is really needed $\rightarrow$ Already known from the bounding boxes
	\item High density $\rightarrow$ most solutions in that area
\end{itemize}

\subsection{Parallel Coordinates}
\begin{itemize}
	\item Parallel lines in PC indicate opposite direction of muscles
	\item Crossing lines indicate similar direction

\end{itemize}


MAYTODOS: (try to answer each one with 2-4 brief sentences, and include citations/notes and questions for me to answer.)
	1. defend why we believe hit-and-run is a good way of viewing densities of distributions.\\
	2. defend why we did the 'every hundredth' point instead of more points//
	3. discuss other approaches to exploring multidimensional spaces//
	4. discuss other ways that people have tried to visualize high dimensional spaces (projections onto n-1 dimensions, etc)//
	explore and discuss which variables could affect the hit-and-run distributions the most if they were stochastically permuted (parametrically) the Jacobian$J^{-T}$, Moment arm matrix $R$, maximal tendon forces ($F_o$).//
	5. briefly explain the bounds of how much the bounding box could feasibly be overestimating the volume of the solution-space polytope. e.g. when would the bounding box be a good approximation? when would the bounding box be misrepresentative of the space?//
	6. fill out the section below on issues with volume computations. Essentially, describe how volume computations have been used in the past, and why we did not use them in our analysis this time. Discuss how, if volume computations were quicker, we could have forseeeably used volume computations. Mention how real-life systems are often way more than 7 dimensions (make a note to cite my vectormap paper), and highlight how the math field feels about the level of computational complexity that hit-and-run computations (ie. would other people agree with us that hit and run is a good approach? if so, cite away!)
	Issues with volume computations:

As realistic musculoskeletal systems has many more muscles, it's important for polytope calculation to be scalable to higher dimensions.

\subsection{Running Time}
However, for each fixed force vector we only have to find a starting point and an orthonormal basis once, and are hence not of concern for the running time. Running one loop of the hit and run algorithm only needs linear time, therefore the method will extend to higher dimesions with only linear factor of additional running time needed.

TODO: add part on how we didn't deal with a dynamical system, but the feasible force space still exists, just with constraints on muscle activations instantaneously, and the momentums present in each of the limbs about their joints.

TODO: discuss how the FAS is an overestimate of the total number of possible activation solutions, because of synergies. Our mathematical approach considers 100 percent independent control of every muscle, with no correlations or inverse correlations between muscle activations. Then slightly discuss how synergies can limit activation capabilities.