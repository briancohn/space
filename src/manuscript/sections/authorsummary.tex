The anatomical layout and interactions among bones, joints and muscles define our ability to produce limb function. How do the brain and spinal cord control our bodies so efficiently, and how does mechanical control evolve over time? Before we can ask these overarching questions, we have to consider the remarkable set of possibilities a controller can use to achieve the same force or movement outcome. The challenge here is adding the physical constraints on the infinite set of possibilities, and enabling visualization of where evolution, learning, and optimization occurs. To this end, we combined approaches from computational geometry and biokinesiology to model endpoint force in a static human finger with seven muscles, and we revealed the true mathematical structure of all feasible solutions. In one of the two visualizations we use to view these structures, one can interactively disable a muscle and see exactly how the central nervous system must change its behavior. Our method was designed to work with an arbitrarily high number of muscles and could be used to predict the underlying altered pathways of learning after an injury.
